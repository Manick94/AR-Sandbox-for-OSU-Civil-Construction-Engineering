\documentclass[10pt]{article}

\usepackage[utf8]{inputenc}
\begin{document}
\title{AR Sandbox Problem Statement}
\begingroup
   \fontsize{20pt}{12pt}\selectfont
    \begin{verbatim}  
AR Sandbox Problem Statement  
    \end{verbatim}  
\endgroup
\begingroup
   \fontsize{15pt}{12pt}\selectfont
    \begin{verbatim}  
By Mark Sprouse  
    \end{verbatim}  
\endgroup
\vfill
\section{Abstract:}

Being able to plan things ahead of time is incredibly valuable.  The easier and more cost-effective planning is, the better it will go and the more options will be available.  This is especially true in regards to construction and civil engineering.  These fields require extensive planning because it is such a detriment when things become delayed or plans have to be dramatically altered.  To aid in this, we will create an Augmented Reality Sandbox which will consist of a simulated environment projected onto sand.  That sand is constantly recorded by a sensor, and the environment will change as the user moves the sand around.  

\newpage
\section{Problem:}

One of the fundamental tasks of any project is planning.  The more planning that happens, the smoother the project ends up going.  Because of this, any improvement in efficiency and quality goes a long way.  People want to get through the planning stages of a project as quickly as possible in order to complete the project as soon as possible, but they dont want to rush things and end up spending more time and money than necessary.  In the field of Construction and Civil Engineering, these potential negatives are incredibly great.   The amount of time wasted and money lost if construction begins and has to be delayed or changed is incredibly large.

\section{Proposed Solution:}

	For a long time now, technology has been an invaluable tool when it comes to planning and preparing.  Now, with augmented reality becoming feasible, all new ways for technology to become useful tools are appearing.  One of these such tools is an Augmented Reality Sandbox.  An Augmented Reality (AR) Sandbox consists of a box of sand, a sensor to detect the levels of the sand, and a projector.  This allows someone to physically change the landscape of the box and have the projection react to that change.  Whether it is a change to the flow of water or the stability of a building, a computer will be able to analyze the ground the simulate what the results will be.

	For the purposes of construction, these sandboxes can be used to help simulate designs and plans in order to get a gauge for the stability and validity of these plans.  Instead of having to create models and perform calculations in order to find out if an idea will work or is worth pursuing, someone can just use the sandbox.

\section{Performance Metrics:}

	In the end, the AR sandbox should project a landscape onto the sand with clearly discerned elements present.  These elements could include roads, bridges, rivers, road barriers, or even homes.  From there, a user should be able to adjust the sand in the box in order to redirect or extend the elements throughout the environment.  As the sand is modified, the projection will react.  

	In addition the sandbox and projected environment, there will be a user-interface that a user can interact with using a mouse that will allow them to place new elements into the environment, delete elements from the environment, record/screenshot the current environment, and alter certain aspects of the simulation possibly including weather and the speed the time is changing.\\

Note: This was written prior to meeting with our client for the first time.


\end{document}

